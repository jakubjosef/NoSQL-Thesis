\pagenumbering{arabic}%start arabic pagination from 1 


\chapter{Úvod}Webové stránky nebo počítačové programy vznikají tak, že je určitý tým lidí naprogramuje v nějakém programovacím jazyce. To už je už prakticky součástí obecného povědomí široké veřejnosti. Avšak neméně důležitou součástí téměř každé aplikace, ať už se jedná o webovou stránku nebo desktopovou aplikaci, je databáze. To je způsob ukládání jakýchkoliv dat, která jsou nezbytná pro fungování aplikace jako celku. Nyní v době obrovského rozvoje a rozmachu propojení informačních technologií se životem obyčejného člověka se značně zvyšují nároky na tyto databáze. Každou sekundou na Internetu přibude obrovské množství dat, které je třeba někde ukládat a zároveň být schopen v nich velmi rychle orientovat. Proto se v posledních letech značně mění náhled na problematiku databázových systémů a můžeme pozorovat odklon od klasických aplikačně složitých SQL serverů k jednodušším řešením. Snížíme tedy složitost databáze,  abychom získali větší rychlost a škálovatelnost. Asi nejznámějším případem je příběh společnosti Facebook, která byla nucena kvůli obrovskému množství dat vytvořit zcela nový databázový systém \cite{cassandra}. Tento databázový systém nebyl založený na principu relačních databází, jednalo se tedy o jeden s prvních případů nasazení nerelační databáze do reálného provozu ve webové aplikaci.  Klasické, mnoho let používané SQL servery jsou typickým příkladem relačních databází, kde jednotlivé entity mají definovány relace na jiné, zatímco  nerelační databáze nic takového neznají a vzhledem k jejich fungování, ani nepotřebují. Jedním ze stěžejních výhod těchto řešení je absence entitového modelu, každá entita tedy může nést libovolné množství jakýchkoliv parametrů různých datových typů. Pro tyto nerelační aplikačně méně komplexní databázová řešení se vžil název NoSQL databáze a bude se jimi zabývat i tato práce.

\section{Cíl, metodika a předpoklady práce}
Hlavním cílem práce je porovnání SQL a NoSQL databází, rychlosti dotazování na jednotlivé databázové servery a představení možností migrace existujících aplikací běžících na klasických SQL serverech. Práce předpokládá základní znalosti z oblasti vývoje webových aplikací a je určena programátorům, kteří mají zájem o použití některé z NoSQL databází místo tradičního SQL řešení. Práce nejprve stručně popíše relační SQL databáze, poté zavede pojem NoSQL databáze, stručně představí aktuální stav v oblasti a existující databázové servery, jejich účel a hlavní výhody. V praktické části se práce zabývá porovnáním relačních SQL databází s novými moderními NoSQL databázemi. Databáze budou porovnávány z hlediska rychlosti zpracování pomocí výkonnostních testů. Testy se budou snažit simulovat situace podobné velmi vytíženým webovým aplikacím, jako je např. objednávka velkého počtu zákazníků najednou, nebo obsluha mnoha uživatelů. Kvůli dobrým možnostem škálování, které NoSQL databáze nabízejí, budou testy také spouštěny na větším počtu databázových serverů, v tzv. clusteru. Cluster bude provozován virtuálně pomocí virtualizační platformy VirtualBox na linuxovém operačním systému Ubuntu a jednotlivé databázové servery budou řešeny pomocí aplikačních kontajnerů Docker. Zároveň bude také otestován výkon databáze pouze na jednom samostatném serveru. Tento server bude provozován lokálně na Mac OS X. Dále budou databáze porovnány podle možností migrace ze zažitých SQL databázových serverů. Na základě těchto testů, bude rozhodnuto zda-li je NoSQL databáze MongoDB vhodnější pro použítí ve webové aplikaci, která by běžně nasadila relační databázi MySQL.

\vspace{0.5cm}
\noindent \emph{Technologie použité v práci:}
\begin{itemize}
\item VirtualBox: virtualizační platforma
\item Vagrant: open source správce virtuálních serverů
\item Ubuntu 12.04: operační systém databázových serverů
\item Docker: nástroj pro tvorbu aplikačních kontajnerů
\item MongoDB: dokumentové orientovaná NoSQL databáze
\item MySQL: open source relační SQL databáze
\item D3.js: framework pro práci s grafy
\item yED: nástroj na tvorbu diagramů a schémat
\end{itemize}

\section{Struktura práce}
Práce se fakticky dělí na dvě základní části. V teoretické části je čtenáři poskytnut lehký úvod do SQL, popsány principy klasických relačních databází a jejich dopady. Dále je v práci představeno NoSQL jako samotný pojem a popsány důvody jeho vzniku. Následuje rozdělení současných NoSQL databázových systémů podle způsobu zacházení s daty. Toto rozdělení se postupem času ustálilo jako to nejlepší. Jednotlivé skupiny databázových systému jsou popsány, je vysvětleno jejich zařazení a zachyceny největší výhody. Téměř každá z těchto skupin byla vytvořena účelně, jako řešení problémů na které relační databáze nestačily. Práce nejpodrobněji popisuje dokumentově orientovanou databázi MongoDB, o které se čtenář mnohé dozví v praktické části této práce. Cílem praktické části je totiž porovnat SQL databázi MySQL právě s MongoDB. V této části práce jsou stručně představeny MySQL a MongoDB servery, včetně ukázek instalace a provozu. Důležitým kritériem každého testování je popis testovacích dat a testovacího prostředí. Těmito dvěma kapitolami práce pokračuje až k samotnému testování. Zde jsou popsány jednotlivé testy, které by měli simulovat situace které v internetovém provozu reálně nastávají. Každý z těchto testů, byl pro srovnání spuštěn i ve virtuálním MongoDB databázovém clusteru. Závěrem práce shrne zjištěné výsledky a nabídne podněty k dalšímu bádání v této oblasti.

\section{Aktuální stav v oblasti NoSQL databází}
Podle aktuálních poznatků v oblasti výkonostních testů databázových serverů by NoSQL databáze, při malém počtu dat, neměly být zásadně rychlejší než klasické SQL databáze. Jejich hlavní síla tkví v jednoduchém škálování výkonu a schopnosti efektivně pracovat s velkým množstvím dat. Aplikačně nejméně komplexní řešení typu \emph{key - value úložišť} se již používají v reálném provozu pro obsluhu velkého počtu jednoduchých dat, například uživatelských sessions \footnote{Sezení uživatele, unikátní řetězec sloužící k identifikaci přihlášeného uživatele.}. Key-Value databáze Redis, vyznačující se velmi vysokou rychlostí zpracování, například dosahuje odezvy kolem 1-2ms při 50 aktivních uživatelích zároveň. Z 10000 požadavků se necelých 80\% zvládlo vyřídit během jedné milisekundy. Tohoto výsledku nebylo dosaženo na nějakém výkonném serveru, ale na obyčejném počítači s dvoujádrovým procesorem a operačním systémem Linux \cite{redisBenchmark}. Toto měření ukazuje, jak rychlá NoSQL databáze může být pokud je aplikačně velmi jednoduchá a dobře navržená. Redis je typickým zástupcem velmi jednoduchých databází, které v podstatě umí pouze uložit hodnotu s nějakým unikátním klíčem, a poté jí zase na jeho základě získat. Tato práce se bude primárně zabývat porovnáním NoSQL databáze MongoDB se SQL databází MySQL. MongoDB je zástupce dokumentově orientovaných databází a je tedy mnohem aplikačně komplexnější databází než již zmíněný Redis. Bývá vnímána jako nejlepší kompromis mezi výkonem a schopnostmi databázového serveru. Podle již provedených testů je několikanásobně rychlejší než MySQL, pokud pracujeme s velkým počtem záznamů (řádově statisíce až miliony) a chceme k nim přistupovat paralelně \cite{nosqlBenchmark}.

Tyto testy byly provedeny na jednom výkonném databázovém serveru a lze očekávat, že při použítí clusteru se výsledky ještě zlepší. Avšak hlavním účelem použití clusteru zůstává vysoká dostupnost a spolehlivost. Jednou z dalších výhod těchto nových technologických řešení je kvalitní návrh, který plně podporuje škálování. To znamená, že lze velmi dobře dynamicky reagovat na zvětšování uživatelské základny webové aplikace. Pokud nastane situace, kdy je databáze přetížená, stačí jednoduše zvýšit výkon clusteru přidáním dalších databových serverů. Teoreticky lze takto navyšovat rychlost a kapacitu databáze dokonečna. Horizontálně škálovatelný návrh je jednou z hlavních výhod NoSQL databází.

\begin{table}[h]
\centering
	\caption{Výkonostní testy MongoDB vs. MySQL \cite{nosqlBenchmark}}
    \begin{tabular}{ | l | l | l | p{5cm} |}
    \hline
    Databáze & Operace & Počet řádků & Čas odpovědi \\ \hline
    MySQL & Vložení dat & 10000000 & 1130493ms \\ \hline
    MongoDB & Vložení dat & 10000000 & 411121ms \\ \hline
    MySQL & Získání dat & 5000 & 66ms \\ \hline
    MongoDB & Získání dat & 5000 & 2ms \\ \hline
    MySQL & Získání dat & 500000 & 447ms \\ \hline
    MongoDB & Získání dat & 500000 & 3ms \\ \hline
    \end{tabular}
    \label{tab:mongoVsMySQLTests}
\end{table}

\section{Literární rešerše v oblasti}
NoSQL databáze jsou relativně mladou a zajímavou technologií, proto se jimi již zabývalo velké množství domácích i zahraničních vědců ve svých pracích. V České republice byl jedním z prvních Richard Günzl na Vysoké škole ekonomické v Praze, který se ve své práci věnoval hlavně obecné problematice NoSQL databází. Představil samotný pojem, nejznámější NoSQL databáze a používané datavé modely. Günzl také ukázal oblasti webové nebo desktopové aplikace, kde je výhodné NoSQL databáze nasazovat \cite{gunzl}. Velmi zajímavou prací, která defakto navazuje na práci Günzla, zpracoval Martin Petera také z VŠE Praha. Jeho práce se dopodrobna zabývá hlavně NoSQL databází MongoDB, popisuje způsob její instalace, obsluhy a implementaci v aplikacích. Je zde dobře popsán proces \emph{mongod}, který zajišťuje běh samotného MongoDB serveru, včetně všech jeho režimů a parametrů. Dále práce popisuje možnosti dotazování a zpracování výsledků nad touto databází \cite{peteraMongo}. V zahraničí se tématu věnoval například Christof Strauch z univerzity ve Stuttgartu, který na téma NoSQL databází zpracoval obsáhlou diplomovou práci. Práce rozebírá důvody vzniku NoSQL databází, jejich základní principy a modely ukládání dat. Poté rozebere tři základní oblasti těchto databází, key-value úložiště, sloupcově-orientované a dokumentové databáze. Práce popíše nejznámější databázové servery v každé z těchto oblastí. Tato práce ale vznikala již od roku 2010 a díky velmi rychlému a dynamickému vývoji v oblasti, obsahuje některé zastaralé nebo překonané pojmy nebo principy \cite{strauchNosql}. Další prací českého studenta je práce s prostým názvem \emph{NoSQL databáze} Tomáše Panyka z Jihočeské univerzity v Českých Budějovicích. Jeho práce kromě obvyklého představení NoSQL databází, porovnává MySQL s Redisem. Redis se jako zástupce úložišť typu klíč - hodnota v tomto porovnání ukázal jako nesrovnatelně rychlejší \cite{panykoNosql}. Tento výsledek se dal vzhledem k principům SQL očekávat a dle mého názoru jsou porovnávány nesprávně vybrané technologie. Redis a MySQL jsou totiž zacíleny na správu úplně odlišných typů dat. Porovnání, kterým se bude zabývat tato práce, tedy porovnání NoSQL databáze MongoDB s SQL databází MySQL se mi jeví přijatelnější, protože oblast jejich nasazení v aplikaci je velmi podobná.
